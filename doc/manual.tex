\documentclass{article}
\usepackage{graphicx} % Required for inserting images
\usepackage[czech]{babel}

\title{PCAP NetFlow v5 exportér}
\author{Jakub Antonín Štigler (xstigl00)}
\date{\today}

\begin{document}

\maketitle

\newpage

\tableofcontents

\newpage

\section{Úvod}
Cílem projektu je vytvořit funkční sondu pro extrakci jednosměrných tokových
dat z pcap souborů. Sonda je zaměřená pouze na TCP packety a exportuje pomoci
UDP ve formátu NetFlow v5, což znamená že zpracovává jen packety přenášené
pomocí IPv4, protože NetFlow v5 nepodporuje IPv6.

Úkolem sondy je agregovat informace o packetech do toků a tyto informace poslat
dál. Tok je síťová komunikace mezi konkrétními počítači kde posílané data spolu
souvisí. Toky se od sebe navzájem rozlišují v tom které počítače si posílají
informace (IP adresy), jaké aplikace spolu komunikují (porty) a tím že
související data se posílají v blízké době. Toky můžou být jednosměrné nebo
obousměrné. V obousměrných tocích se v toku shlukuje komunikace z obou stran,
v jednosměrných pak je jen komunikace jedním směrem a toků je tím více. Formát
NetFlow v5 podporuje jen jednosměrné toky, a proto p2nprobe také podporuje jen
jednosměrné toky.

\section{Návrh}
Pro zjednodušení práce jsem se rozhodl obalovat různé části knihoven s C
rozhraním (např. třída \verb|UniqueFd| pro zabalení deskriptoru).

Dále jsem program rozdělil do několika částí/komponent pro řešení jednotlivých
částí problému:
\begin{itemize}
    \item Zpracování argumentů (adresář \textit{src/cli/}).
    \item Čtení pcap souborů (adresář \textit{src/pcap/}).
    \item Parsování packetů. (adresář \textit{src/parser/}).
    \item Agregace do toků a ukládání dat (adresář \textit{src/storage/}).
    \item Export (adresář \textit{src/out/})
\end{itemize}

\section{Popis implementace}
Projekt jsem se rozhodl implementovat v programovacím jazyce C++, protože už s
ním mám zkušenosti, nabízí spoustu užitečných funkcí a je pro tento projekt
vhodný i protože je kompilovaný a umožňuje tak zpracovávat toky na vysoké
rychlosti.

Pro sestavování využívám cmake.

Využívám verze C++20, takže pro kompilaci je potřeba g++ s verzí minimálně 13.
Projekt funguje i při kompilaci s clang++ 18. Aby kompilace na serveru merlin
proběhla úspěšně, tak mám kompiler explicitně nastavený na \verb|g++-14.2|,
protože jinak by se implicitně použil \verb|g++| s verzí 11.5. Protože výsledný
program potřebuje novější verzi standartní knihovny než která se na merlinovi
najde automaticky, nastavil jsem že se standartní C++ knihovna linkuje
staticky.

Při implementaci jsem se pro propagaci chyb rozhodl využít vyjímek.

Vstupní funkce \verb|main| se nachází v souboru \textit{src/main.cpp}.

Hlavní smyčka programu se odehrává v implementaci metody \verb|run| ve třídě
\verb|Pipeline| (implementováno v souborech \textit{src/pipeline.hpp} a
\textit{src/pipeline.cpp}). \verb|Pipeline| umožňuje použití různých způsobů
exportování za pomocí rozhraní \verb|Exporter| (definováno v
\textit{src/out/exporter.hpp}).

\subsection{Zpracování vstupních argumentů}
Vstupní argumenty se zpracovávají pomocí třídy \verb|Args| implementované v
souborech \textit{src/cli/args.hpp} a \textit{src/cli/args.cpp}. Načtená data
se s načítáním rovnou validují a případně se vavolá vyjímka s uživatelsky
přívětivou chybovou hláškou.

\subsection{Čtení pcap souborů}
Pro čtení z pcap souborů využívám knihovny \textit{libpcap}. Část funkcionality
jsem obalil do vlastních funkcí nebo tříd protože \textit{libpcap} má rozhraní
v jazyce C a nevyužívá různých užitečných konstrukcí jazyka C++ jako třeba
třídy, destruktory nebo vyjímky.

Funkcionalitu knihovny jsem obalil v souborech které se nachází v adresáři
\textit{src/pcap/}.

Pro čtení packetů využívám funkce \verb|pcap_next_ex|, která vrací packety
jeden po jednom a pak jsou postupně zpracovávány.

\subsection{Parsování packetů}
Data z přečteného packetu se uloží do struktury \verb|Packet|, která je
definovaná v souboru \textit{src/storage/packet.hpp}. Čas zaznamenání packetu
převádím do C++ reprezentace času protože standartní knihovna obsahuje spoustu
užitečných funkcí pro práci s časem.

Většina kódu pro parsování packetů se nachází v adresáři \textit{src/parser/}.
Parsování různých vrstev (ethernet, ipv4, tcp) je rozděleno do různých souborů.
Parsování většinou probíhá jen použitím \verb|reinterpret_cast| pro
interpretaci dat jako struktura hlavičky daného protokolu. Data posílané přes
internet mají většinou pořadí bytů big-endian, ale počítače pracují s pořadím
little-endian. Proto je nutné tyto data převádět. Pro převod jsem vytvořil
třídu \verb|Be| (implementovaná v souboru \textit{src/endian.hpp}), díky které
jde informaci o jiném pořadí bytů zakódovat přímo do datového typu. Hodnota je
pak automaticky převedena do správného pořadí bytů vždy když je potřeba s ní
pracovat a není nutné na to už myslet.

\subsubsection{Informace sbírané z packetů}
\begin{itemize}
    \item Počet bytů ve vrstvě L3.
    \item Zdrojová a cílová IP adresa.
    \item Protokol transportní vrstvy (vždy TCP, protože ostatní packety jsou
          zahazovány).
    \item TOS z IP vrstvy (Type Of Service).
    \item Zdrojový a cílový port.
    \item TCP příznaky.
    \item Čas zachycení.
\end{itemize}

\subsection{Skládání packetů do toků}
Packety jsou skládány do toků podle klíče složeného z klíčových informací které
jsou v jednom toku vždy stejné. Jsou to informace:
\begin{itemize}
    \item Zdrojová a cílová adresa.
    \item Zdrojový a cílový port.
    \item Protokol (vždy TCP).
    \item IP TOS (Type Of Service).
\end{itemize}
Klíč je definovaný v souboru \textit{src/storage/flow\_key.hpp}.

Toky jsou shlukovány za pomocí hashovací tabulky \verb|std::unordered_map| ze
standartní knihovny. Shlukování je implementováno ve třídě \verb|FlowCache|,
která je v souborech \textit{src/storage/flow\_cache.hpp} a
\textit{src/storage/flow\_cache.cpp}.

Pro aktivní timeout se všechny toky také ukládají do fronty v pořadí ve kterém
přišly. Po každém zpracovaném packetu se ze začátku fronty exportují všechny
toky kterým vypršel aktivní timeout.

Pasivní timeout je řešený vždy při přidávání packetu do existujícího toku.
Pokud vypršel pasivní timeout, starý tok je exportován a je vytvořen nový.

\subsubsection{Informace sbírané v tocích}
Tok je definován v souboru \textit{src/storage/flow.hpp}. Obsahuje informace:
\begin{itemize}
    \item Zdrojová a cílová IP adresa.
    \item Počet poslaných packetů.
    \item Množství poslaných dat na vrstvě L3.
    \item Čas zachycení prvního packetu v toku.
    \item Čas zachycení posledního packetu v toku.
    \item Zdrojový a cílový port.
    \item Kumulativní OR TCP příznaků.
    \item Protokol použitý v toku.
    \item IP TOS (Type Of Service)
\end{itemize}

\subsection{Export do NetFlow v5}
Export je z většiny implementovaný v adresáři \textit{src/out/}. Kromě exportu
do NetFlow v5 je zde implementovaný i export do textu, který jsem využíval pro
testovací účely.

Export pro NetFlow v5 zastřešuje třída \verb|NetFlowV5| implementovaná v
souborech \textit{src/out/net\_flow\_v5.hpp} a
\textit{src/out/net\_flow\_v5.cpp}. Třída přijímá toky a čeká dokud až jich
bude mít dostatek. Když má maximální počet toků co jde poslat najednou tak toky
odešle za pomocí třídy \verb|UdpClient| implementované v souborech
\textit{src/udp\_client.hpp} a \textit{src/udp\_client.cpp}.

Data jsou převedena do NetFlow v5 pomocí struktur
\verb|NetFlowV5Header| a \verb|NetFlowV5Record|. Zde se opět hodí třída
\verb|Be| pro převod dat na big-endian.

\subsubsection{UdpClient}
\verb|UdpClient| je třída obalující práci se sokety. Dokáže za pomocí DNS
vyhledat IP adresu serveru na který se posílají data, vytvořit socket a
následně data odesílat.

Pro správu deskriptoru pro soket jsem vytvořil třídu \verb|UniqueFd|
implementovanou v souboru \textit{src/unique\_fd.hpp}.

\section{Návod na použití}
Pro vypsání stručné nápovědy se dá program spustit jako:
\begin{verbatim}
    p2nprobe -h
\end{verbatim}

Pro čtení pcap souboru a export se program spustí jako:
\begin{verbatim}
    p2nprobe <adresa>:<port> <pcap soubor> [-a <aktivní timeout>]
             [-i <neaktivní timeout>]
\end{verbatim}
\verb|adresa| může být libovolná IPv4 nebo IPv6 adresa, nebo adresa která musí
být přeložená pomocí DNS. Jedná se o adresu serveru na který budou pomocí UDP
exportovány informace o tocích.

\verb|port| je číslo v rozmezí 0 - 65535. Jde o port serveru na který se
odesílají informace o tocích.

\verb|aktivní timeout| je aktivní timeout v celých sekundách. Parametr není
povinný a implicitně má hodnotu \verb|60|. Když uplyne tento čas od zachycení
prvního packetu z toku, tak je tok exportován.

\verb|neaktivní timeout| je neaktivní timeout v sekundích. Parametr není
povinný a implicitně má hodnotu \verb|60|. Když by měl být packet přiřazen do
toku, ale od posledního packetu v toku uplynula tato doba, tok je exportován
a nový packet založí nový tok.

Argumenty můžou být v libovolném pořadí.

\subsection{Příklady použití}
Zobrazení nápovědy:
\begin{verbatim}
    p2nprobe -h
\end{verbatim}

Export toků ze souboru \verb|capture.pcap| na adresu \verb|127.0.0.1| a port
\verb|25565| s implicitními timeouty 60 s:
\begin{verbatim}
    p2nprobe 127.0.0.1:25565 capture.pcap
\end{verbatim}

Export toků ze souboru \verb|ipv4.pcap| na adresu \verb|localhost| a port
\verb|25565| s aktivním timeoutem nastaveným na 300 s a neaktivním timeoutem
implicitně nastaveným na 60 s.
\begin{verbatim}
    p2nprobe ipv4.pcap -a 300 localhost:25565
\end{verbatim}

\section{Testování}
Pro testování jsem vytvořil jednoduchý přijímač pro NetFlow v5 (n2tcap)
implementovaný v adresáři \textit{n2tcap}, který vypisuje přijaté data na
standartní výstup. Porovnával jsem pak výstup při použití p2nprobe a softflowd.
Z výstupu jsem usoudil že program funguje správně ikdyž výstupy nebyly totožné.
Výstup softflowd měl některé toky spojené dohromady protože z mě neznámého
důvodu nerespektoval aktivní timeout. Další rozdíl byl že softflowd nevyplňoval
políčko pro IP TOS.

Při testování jsem využíval address sanitizer a neukazuje žádné chyby.

Pro detekci častých chyb a konzistentní formátování kódu jsem využil nástrojů
clang-format, cppcheck a clang-tidy.

\end{document}
